\documentclass[11pt]{beamer}
\usetheme{Warsaw}
\usepackage[utf8]{inputenc}
\usepackage[french]{babel}
\usepackage[T1]{fontenc}
\usepackage{amsmath}
\usepackage{amsfonts}
\usepackage{amssymb}
\usepackage{graphicx}
\author{Théophile \textsc{Bastian}, Noémie \textsc{Cartier}, Nathanaël \textsc{Courant}}
\title{Système Digital~: horloge virtuelle}
%\setbeamercovered{transparent} 
\setbeamertemplate{navigation symbols}{} 
%\logo{} 
%\institute{} 
\date{26 janvier 2016} 
%\subject{} 

\usepackage{my_listings}


\begin{document}

\begin{frame}
\titlepage
\end{frame}

%\begin{frame}
%\tableofcontents
%\end{frame}

%%%%%%%%%%%%%%%%%%%%%%%%%%%%%%%%%%%%%%%%%%%%%%%%%%%%%
\section*{Vue d'ensemble}

\begin{frame}{Vue d'ensemble}
\begin{itemize}
\item Processeur généré en Python
\item Netlist compilée vers C
\item Sortie binaire envoyée sur une interface graphique
\item Architecture inspirée d'ARM
\end{itemize}
Résultat~:
\begin{itemize}
\item \textit{Rapide}~: 9.25$j.s^{-1}$~; 11.5$j.s^{-1}$ en faisant fondre un processeur
\item $1.6MHz$~; $2MHz$ en faisant fondre un processeur
\end{itemize}
\end{frame}

\begin{frame}
\setcounter{tocdepth}{1} % Limite la TOC à sections
\tableofcontents
\end{frame}

%%%%%%%%%%%%%%%%%%%%%%%%%%%%%%%%%%%%%%%%%%%%%%%%%%%%%
\section{Compilateur de netlists}
\begin{frame}{Compilateur de netlists}
Fonctionnement~:
\begin{itemize}
\item{Nappes de fils~: entiers 64 bits}
\item{Tri topologique des équations}
\item{Produire le code pour chaque équation}
\end{itemize}

Optimisations~:
\begin{itemize}
\item{Opérations \verb!SLICE!, \verb!SELECT!, \verb!CONCAT! quand on
    peut}
\item{Opérations bit-à-bit sur les nappes si possible}
\item{\og{}\verb!-!\fg{} unaire pour étendre un fil en une nappe ayant cette
    valeur}
\end{itemize}
\end{frame}

%%%%%%%%%%%%%%%%%%%%%%%%%%%%%%%%%%%%%%%%%%%%%%%%%%%%%
\section{Assembleur}

\subsection{Opérations}

\begin{frame}{Opérations de l'assembleur}

Les opérations supportées par notre processeur~:

\begin{itemize}
\item opérations arithmétiques d'additions et de soustraction~;

\item opérations binaires classiques~;

\item déplacements d'une valeur ou de son opposé d'un registre vers un autre~;

\item accès RAM (en lecture ou en  écriture)~;

\item saut vers un label \lstc{JMP}.
\end{itemize}

Chacune peut-être assortie d'une conditionnelle en fonctions des flags placés par l'opération précédente.

\end{frame}

\subsection{Opcodes}

\begin{frame}{Les opcodes}

Forme de l'opcode produit~:

\begin{tabular}{|c|c|l|}
\hline
\textbf{Bits} & \textbf{Longueur} & \textbf{Contenu}\\
\hline
1 -- 4 & 4 & Condition d'exécution \\
5 -- 8 & 4 & Code de l'instruction \\
9 & 1 & Écrire le résultat~? \\
10 & 1 & Utiliser le \textit{carry bit}~?  \\
11 & 1 & Écrire les flags~? \\
12 -- 15 & 4 & Registre de destination \\
16 -- 19 & 4 & $op_1$ \\
20 & 1 & Remplacer $op_1$ par $0\cdots 0$~? \\
21 -- 46 & 25 & $op_2$ \\ \hline
\end{tabular}

\end{frame}


%%%%%%%%%%%%%%%%%%%%%%%%%%%%%%%%%%%%%%%%%%%%%%%%%%%%%
\section{Horloge}

\subsection{Quartz}

\begin{frame}{Quartz}
\begin{itemize}
\item \textit{Pipe} sur le processeur
\item Initialise au temps voulu
\item Horloge synchrone~: \alert{1} une fois par seconde (incrément), \alert{0} sinon
\item Deux programmes distincts~:
	\begin{itemize}
	\item Synchrone~: code python (plus simple, moins rapide)
	\item Rapide~: code C~; initialise, \lstc{while(1) \{ putchar(0); \}}
	\end{itemize}
\item Gain Python $\rightarrow$ C~: $\times 3$
\end{itemize}
\end{frame}

\subsection{Programme}
\begin{frame}{Programme}
\begin{itemize}
\item{Boucle des secondes déroulée}
\item{Une instruction sur deux~: mettre à jour l'affichage}
\item{L'autre~: calculer l'affichage de la minute suivante}
\item{Exactement 2 instructions par seconde simulée (optimal)}
\item{Gestion des années bissextiles, y compris exceptions}
\item{Initialisation~: difficulté}
\end{itemize}
\end{frame}


%%%%%%%%%%%%%%%%%%%%%%%%%%%%%%%%%%%%%%%%%%%%%%%%%%%%%
\section{Processeur}



%%%%%%%%%%%%%%%%%%%%%%%%%%%%%%%%%%%%%%%%%%%%%%%%%%%%%
\section{Interface graphique}

\begin{frame}{Interface graphique}
\begin{itemize}
\item C++, avec Qt
\item Communication~: lit 16 caractères par rafraîchissement (\lstc{stdin})
\item Un chiffre = un caractère~; un bit = un segment
\item Affichage à 30 FPS
\end{itemize}

\noindent Optimisations~:
\begin{itemize}
\item \alert{Deux threads}~: affichage, lecture de l'entrée
\item Première idée~: quand demandé, lire 16 caractères puis \lstc{fflush}
\item Utilisé~: lire en continu, donner les 16 derniers
\item Trop lent~: ignorer \alert{12 cycles sur 13} ($\in \mathcal{P}$)
\end{itemize}
\end{frame}

\section*{}

\begin{frame}{Conclusion}
\begin{center}\Huge
Merci de votre attention~!
\end{center}
\end{frame}

\end{document}