\documentclass[11pt]{beamer}
\usetheme{Warsaw}
\usepackage[utf8]{inputenc}
\usepackage[french]{babel}
\usepackage[T1]{fontenc}
\usepackage{amsmath}
\usepackage{amsfonts}
\usepackage{amssymb}
\usepackage{graphicx}
\author{Théophile \textsc{Bastian}, Noémie \textsc{Cartier}, Nathanaël \textsc{Courant}}
\title{Système Digital~: horloge virtuelle}
%\setbeamercovered{transparent} 
\setbeamertemplate{navigation symbols}{} 
%\logo{} 
%\institute{} 
\date{26 janvier 2016} 
%\subject{} 

\usepackage{my_listings}


\begin{document}

\begin{frame}
\titlepage
\end{frame}

%\begin{frame}
%\tableofcontents
%\end{frame}

%%%%%%%%%%%%%%%%%%%%%%%%%%%%%%%%%%%%%%%%%%%%%%%%%%%%%
\section*{Vue d'ensemble}

\begin{frame}{Blah}
Blah!!1
\end{frame}


\begin{frame}
\tableofcontents
\end{frame}

%%%%%%%%%%%%%%%%%%%%%%%%%%%%%%%%%%%%%%%%%%%%%%%%%%%%%
\section{Compilateur de netlists}
\begin{frame}{Compilateur de netlists}
Fonctionnement~:
\begin{itemize}
\item{Nappes de fils~: entiers 64 bits}
\item{Tri topologique des équations}
\item{Produire le code pour chaque équation}
\end{itemize}

Optimisations~:
\begin{itemize}
\item{Opérations \verb!SLICE!, \verb!SELECT!, \verb!CONCAT! quand on
    peut}
\item{Opérations bit-à-bit sur les nappes si possible}
\item{\og{}\verb!-!\fg{} unaire pour étendre un fil en une nappe ayant cette
    valeur}
\end{itemize}
\end{frame}

%%%%%%%%%%%%%%%%%%%%%%%%%%%%%%%%%%%%%%%%%%%%%%%%%%%%%
\section{Assembleur}



%%%%%%%%%%%%%%%%%%%%%%%%%%%%%%%%%%%%%%%%%%%%%%%%%%%%%
\section{Horloge}

\subsection{Quartz}

\subsection{Programme}
\begin{frame}{Programme}
\begin{itemize}
\item{Boucle des secondes déroulée}
\item{Une instruction sur deux~: mettre à jour l'affichage}
\item{L'autre~: calculer l'affichage de la minute suivante}
\item{Exactement 2 instructions par seconde simulée (optimal)}
\item{Gestion des années bissextiles, y compris exceptions}
\item{Initialisation~: difficulté}
\end{itemize}
\end{frame}


%%%%%%%%%%%%%%%%%%%%%%%%%%%%%%%%%%%%%%%%%%%%%%%%%%%%%
\section{Processeur}



%%%%%%%%%%%%%%%%%%%%%%%%%%%%%%%%%%%%%%%%%%%%%%%%%%%%%
\section{Interface graphique}

\begin{frame}{Interface graphique}
\begin{itemize}
\item C++, avec Qt
\item Communication~: lit 16 caractères par rafraîchissement (\lstc{stdin})
\item Un chiffre = un caractère~; un bit = un segment
\item Affichage à 30 FPS
\end{itemize}

\noindent Optimisations~:
\begin{itemize}
\item \alert{Deux threads}~: affichage, lecture de l'entrée
\item Première idée~: quand demandé, lire 16 caractères puis \lstc{fflush}
\item Utilisé~: lire en continu, donner les 16 derniers
\item Trop lent~: ignorer \alert{12 cycles sur 13} ($\in \mathcal{P}$)
\end{itemize}
\end{frame}

\end{document}