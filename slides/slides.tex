\documentclass[11pt]{beamer}
\usetheme{Warsaw}
\usepackage[utf8]{inputenc}
\usepackage[french]{babel}
\usepackage[T1]{fontenc}
\usepackage{amsmath}
\usepackage{amsfonts}
\usepackage{amssymb}
\usepackage{graphicx}
\author{Théophile \textsc{Bastian}, Noémie \textsc{Cartier}, Nathanaël \textsc{Courant}}
\title{Système Digital~: horloge virtuelle}
%\setbeamercovered{transparent} 
\setbeamertemplate{navigation symbols}{} 
%\logo{} 
%\institute{} 
\date{26 janvier 2016} 
%\subject{} 
\begin{document}

\begin{frame}
\titlepage
\end{frame}

%\begin{frame}
%\tableofcontents
%\end{frame}

%%%%%%%%%%%%%%%%%%%%%%%%%%%%%%%%%%%%%%%%%%%%%%%%%%%%%
\section*{Vue d'ensemble}

\begin{frame}{Blah}
Blah!!1
\end{frame}


\begin{frame}
\tableofcontents
\end{frame}

%%%%%%%%%%%%%%%%%%%%%%%%%%%%%%%%%%%%%%%%%%%%%%%%%%%%%
\section{Compilateur de netlists}


%%%%%%%%%%%%%%%%%%%%%%%%%%%%%%%%%%%%%%%%%%%%%%%%%%%%%
\section{Assembleur}

\begin{frame}{Opérations de l'assembleur}

Les opérations supportées par notre processeur~:

\begin{itemize}
\item opérations arithmétiques d'additions et de soustraction (avec et sans retenue), en retenant ou non le résultat~;

\item opérations binaires classiques, en retenant ou non le résultat~;

\item déplacements d'une valeur ou de son opposé d'un registre vers un autre~;

\item accès RAM (en lecture ou en  écriture)~;

\item le saut vers un label \lstc{JMP}.
\end{itemize}

Chacune peut-être assortie d'une conditionnelle en fonctions des flags placés par l'opération précédente.

\end{frame}

%%%%%%%%%%%%%%%%%%%%%%%%%%%%%%%%%%%%%%%%%%%%%%%%%%%%%
\section{Horloge}

\subsection{Quartz}

\subsection{Programme}



%%%%%%%%%%%%%%%%%%%%%%%%%%%%%%%%%%%%%%%%%%%%%%%%%%%%%
\section{Processeur}



%%%%%%%%%%%%%%%%%%%%%%%%%%%%%%%%%%%%%%%%%%%%%%%%%%%%%
\section{Interface graphique}

\end{document}